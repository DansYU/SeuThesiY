\documentclass[algorithmlist, figurelist,tablelist, nomlist,phd]{config/seuthesiY}

%===============================================================================
% 采用的编译方式为  XELATEX -→  BIBER -→    XELATEX -→    XELATEX
% 为了加快字体的缓存效率 需要在命令行运行 fc-cache
% 对于学位论文需要标注【硕博】
%===============================================================================

% %biblatex宏包的参考文献数据源加载方式
\addbibresource[location=local]{config/seuthesiY.bib}

\begin{document}
%===============================================================================
\categorynumber{000} % 分类采用《中国图书资料分类法》
\UDC{000}            %《国际十进分类法UDC》的类号
\secretlevel{公开}    %学位论文密级分为"公开"、"内部"、"秘密"和"机密"四种
\studentid{130623}   %学号要完整,前面的零不能省略。
\title{灵犀一指GNSS心法}{灵犀一指 \rotatebox{270}{GNSS} 心法}{灵犀一指}{灵犀一指}{The theory of powerful fingers}{powerful fingers}
\author{陆小凤}{Phoenix Land, Jr.}
\advisor{夜帝}{教授}{King Night}{Prof.}
\coadvisor{楚留香}{副教授}{Perfume Tsu}{Associate Prof.} % 没有% 可以不填
\degreetype{武学博士}{Doctor of kung fu} % 详细学位名称
\thesisform{应用研究} % 包括应用研究、调研报告、规划、产品开发、案例分析、项目管理、文学艺术作品、其它。非专业型硕士可忽略
\major{内功}
\submajor{内功心法}
\defenddate{\today}
\authorizedate{\today}
\committeechair{夜帝}
\reviewer{张三丰}{黄药师}
\department{东南大学武学院}{School of kung fu}
\seuthesisthanks{本课题的研究获郭靖-黄蓉降龙基金、杨过-小龙女黯然销魂基金以及郭襄的倚天基金资助}
\makebigcover
\makecover
\begin{abstract}{武功,心法,内功,灵犀一指}
    灵犀一指是一种非常厉害的武功。
\end{abstract}

\begin{englishabstract}{kung fu, theory, fundamental kung fu, powerful fingers}
    powerful fingers is a kind of powerful kung fu.
\end{englishabstract}

\setnomname{术语与符号约定}
\tableofcontents
\listofothers
%===============================================================================


\mainmatter

\chapter{绪论}
\section{研究背景}
天下武功,无坚不破,唯快不破。灵犀一指属于快而非坚之武学。


\section{本论文的工作}
本论文的研究对象为灵犀一指,着重研究其中的内功心法。
%正文内容,引用参考文献
详见文献\cite{Peebles2001-100-100}\parencite{Babu2014--}
参考文献\cite[见][49页]{于潇2012-1518-1523}\parencite[见][49页]{Babu2014--}
硕士论文\cite{zhouGPS2015},博士论文\cite{余勇1998--}
大小写\cite{liu_statistical_2017}

\nomenclature{PF}{powerful fingers}
如图\ref{lxfbook}所示。

\begin{figure}
    \centering
    \includegraphics[width=.6\textwidth]{lxfbook.jpg}
    \caption{陆小凤传奇\label{lxfbook}}
\end{figure}
\nomenclature{KF}{kung fu}


\backmatter
%打印参考文献表
\thesisbib

\appendix


\resume{作者攻读博士学位期间的研究成果}
\begin{flushleft}
    {\bfseries \large 发表的论文}\\ \relax
    [1] 第一作者,“灵犀一指:理论与应用”, 武侠学报,
    2015年5月。\\
\end{flushleft}


\end{document}
